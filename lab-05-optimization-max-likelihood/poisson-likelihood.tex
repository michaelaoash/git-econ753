\documentclass{beamer}\usepackage{pgfpages} \mode<beamer>{ \usetheme{Warsaw}
\setbeamertemplate{footline}[page number]
\setbeamertemplate{navigation symbols}{}} 
\newcommand{\cov}{\text{cov}}

\usepackage{lmodern}

\usepackage{dcolumn}
\newcolumntype{.}{D{.}{.}{-1}}


\title{ Applied Econometrics }
\subtitle{ Labor Econometrics}
\author{Michael Ash}
\institute{Applied Econometrics}
\date{}

\begin{document}

\begin{frame}
  Poisson probability formula
\begin{displaymath}
  \Pr (Y = k) = \frac{e^{-\lambda} \cdot \lambda^k  }{k!}
\end{displaymath}

\begin{itemize}
\item This is the probability of exactly $k$ (non-negative integer)
  occurrences of a random count variable $Y$.
\item There is only one parameter, $\lambda$, the ``arrival rate'' or
  the Expected (average) number of occurrences, which makes the
  expression easy to manipulate.
\item It is the limiting case of a binomial process when the number of
  trials $n$ is high and the probability $p$ is low. The sole
  parameter $\lambda = np$ is the expected count. 
\item It is not surprising that $E(Y)=\lambda$.  It may be more surprising that, if it's really Poisson, also $\text{var}(Y)=\lambda$ 
\end{itemize}

\end{frame}

\begin{frame}[fragile]
  We work with the Kiefer strike duration data recorded between 1968
  and 1976. Each of the 62 observations is a strike for which the main
  variable of interest (dependent variable) is the duration of the
  strike in days.  Other variables include the year of the strike and
  an industrial production index for the year. The logic of the model is that
  strikes will be shorter when industrial production is high because
  firms have strong incentives to settle and get back to making
  profits.

\begin{verbatim}
 year strike_duration       ip
 1968               7  0.01138
 1968               9  0.01138
 1968              13  0.01138
 1968              14  0.01138
 1968              26  0.01138
 1968              29  0.01138
 1968              52  0.01138
 1968             130  0.01138
 1969               9  0.02299
 1969              37  0.02299
 1969              41  0.02299
 1969              49  0.02299
 1969              52  0.02299
 1969             119  0.02299
\end{verbatim}
\end{frame}


\begin{frame}

How likely were we to see the exact sample we saw?
\begin{displaymath}
  \Pr \left( Y_1= y_1, Y_2= y_2, \ldots Y_n= y_n | \lambda  \right)
\end{displaymath}

\begin{displaymath}
  \Pr \left( Y_1= y_1 | \lambda \right) \cdot
  \Pr \left( Y_2= y_2 | \lambda \right) \cdot
  \cdots
  \Pr \left( Y_n= y_n | \lambda \right) 
\end{displaymath}


\begin{displaymath}
  \frac{e^{-\lambda} \cdot \lambda^{y_1}}{{y_1}!}  \cdot
  \frac{e^{-\lambda} \cdot \lambda^{y_2}}{{y_2}!}  \cdot
  \cdots
  \frac{e^{-\lambda} \cdot \lambda^{y_n}}{{y_n}!}  
\end{displaymath}

Reminder the little $y_i$ are data observed in the world.
\end{frame}



\begin{frame}
\begin{displaymath}
\mathcal{L} = \frac{e^{-\lambda} \cdot \lambda^{y_1}}{{y_1}!}  \cdot
  \frac{e^{-\lambda} \cdot \lambda^{y_2}}{{y_2}!}  \cdot
  \cdots
  \frac{e^{-\lambda} \cdot \lambda^{y_n}}{{y_n}!}  
\end{displaymath}

Taking log converts multiplication problems into additional problems

\begin{displaymath}
\ln \mathcal{L} = \ln \frac{e^{-\lambda} \cdot \lambda^{y_1}}{{y_1}!}  +
  \ln \frac{e^{-\lambda} \cdot \lambda^{y_2}}{{y_2}!}  +
  \cdots
  \ln \frac{e^{-\lambda} \cdot \lambda^{y_n}}{{y_n}!}  
\end{displaymath}


\end{frame}

\begin{frame}
  

\begin{displaymath}
\ln \mathcal{L} = \ln \frac{e^{-\lambda} \cdot \lambda^{y_1}}{{y_1}!}  +
  \ln \frac{e^{-\lambda} \cdot \lambda^{y_2}}{{y_2}!}  +
  \cdots
  \ln \frac{e^{-\lambda} \cdot \lambda^{y_n}}{{y_n}!}  
\end{displaymath}

Zoom in on one of the elements:

\begin{displaymath}
  \ln \mathcal{L}_1 = \ln \frac{e^{-\lambda} \cdot \lambda^{y_1}}{{y_1}!}
  =  \ln e^{-\lambda} + \ln \lambda^{y_1}  - {\ln {y_1}!}  =
\end{displaymath}
\begin{displaymath}
 -\lambda + y_1 \ln \lambda - \ln (y_1) - \ln (y_1 - 1) - \ln (y_1 - 2) - \cdots - \ln 1
\end{displaymath}

Zoom back out (there are $n$ of these expressions to add up):

\begin{displaymath}
  \ln \mathcal{L} = -n \lambda + (y_1 + y_2 + \cdots + y_n) \ln \lambda 
\end{displaymath}
\begin{align}
 - \ln (y_1) - \ln (y_1 - 1) - \ln (y_1 - 2) - \cdots - \ln 1 \nonumber\\
 - \ln (y_2) - \ln (y_2 - 1) - \ln (y_2 - 2) - \cdots - \ln 1 \nonumber\\
 - \ln (y_n) - \ln (y_n - 1) - \ln (y_n - 2) - \cdots - \ln 1 \nonumber
\end{align}
\end{frame}

\begin{frame}
  The second batch of stuff changes the level of  $\ln \mathcal{L}$ but does not depend on $\lambda$.  Only the first batch does:
\begin{displaymath}
  \ln \mathcal{L} = -n \lambda + (y_1 + y_2 + \cdots + y_n) \ln \lambda 
\end{displaymath}
\begin{align}
  - \ln (y_1) - \ln (y_1 - 1) - \ln (y_1 - 2) - \cdots - \ln 1 \nonumber\\
 - \ln (y_2) - \ln (y_2 - 1) - \ln (y_2 - 2) - \cdots - \ln 1 \nonumber\\
 - \ln (y_n) - \ln (y_n - 1) - \ln (y_n - 2) - \cdots - \ln 1 \nonumber
\end{align}


\begin{displaymath}
  \max_\lambda   \ln \mathcal{L} = -n \lambda + (y_1 + y_2 + \cdots + y_n) \ln \lambda 
\end{displaymath}


\begin{displaymath}
  -n  + \frac{1}{\hat{\lambda}} (y_1 + y_2 + \cdots + y_n) \equiv 0
\end{displaymath}

\begin{displaymath}
  \hat{\lambda} = (y_1 + y_2 + \cdots + y_n) / n
\end{displaymath}


\end{frame}



\begin{frame}
  But now let's revisit that with slightly different syntax. Let
  $\beta = \ln \lambda$ and maximize with respect to $\beta$


\begin{displaymath}
  \max_\lambda   \ln \mathcal{L} = -n \lambda + (y_1 + y_2 + \cdots + y_n) \ln \lambda 
\end{displaymath}


\begin{displaymath}
  \max_{\ln \lambda}   \ln \mathcal{L} = -n e^{\ln \lambda} + (y_1 + y_2 + \cdots + y_n) \ln \lambda 
\end{displaymath}


\begin{displaymath}
  \max_{\beta}   \ln \mathcal{L} = -n e^{\beta} + (y_1 + y_2 + \cdots + y_n) \beta
\end{displaymath}


\begin{displaymath}
  -n e^{\hat{\beta}}  + (y_1 + y_2 + \cdots + y_n) \equiv 0
\end{displaymath}

\begin{displaymath}
  e^{\hat{\beta}} = (y_1 + y_2 + \cdots + y_n)/n
\end{displaymath}

\begin{displaymath}
  \hat{\beta} = \ln \left( (y_1 + y_2 + \cdots + y_n)/n \right)
\end{displaymath}
  
\end{frame}

\begin{frame}
  In the code below, \texttt{y} is the full vector of strike durations $y = {7, 9, 13, 14, 26,\ldots}$, \texttt{cons} is the same length and made of 1's.  The last term, which is returned by the function, is the log likelihood, for each observation $-e^{\beta} + y_i \beta$ summed over all observations, or $-ne^{\beta} + \sum y_i \beta $.
\end{frame}  

\begin{verbatim}
## log(lambda) = XB, in this case log(lambda) = B
poissonll <- function(beta) {
    y <- kiefer$strike_duration
    cons <- rep(1, length(kiefer$strike_duration))
    -sum( -exp(beta*cons) + y * beta * cons ) }
\end{verbatim}


This function ``rates'' the quality of the current guess of beta by
returning the (negative) log likelihood for that guess of beta.  We
give this function to R's optimizer and let the optimizer find the
value of beta that maximizes the log likelihood (formally, by
minimizing the negative log likelihood).

\begin{verbatim}
## Use optimizer to find beta, the log lambda that minimizes negative log likelihood
(poissonll.opt  <- optim( par=1,  poissonll ))
## Report lambda
exp(poissonll.opt$par)
\end{verbatim}


\begin{frame}
  This syntax lets us introduce covariates into the Poisson regression.

  Now each observation $i$ has its own ``arrival rate'' $\lambda_i$,
  which we express logged, $\ln \lambda_i$.  Instead of
  $\ln \lambda_i = \beta$, we let
  $\ln \lambda_i = \beta_1 + \beta_2 \text{ip}_i$, where $\text{ip}_i$
  is the industrial production index for the year in which strike $i$
  took place.

  Strike $i$'s duration, which can be any non-negative integer, is drawn from a Poisson distribution in which the expected arrival rate is $\lambda_i$, which depends on the covariate $\text{ip}_i$ in this particular way, $\ln \lambda_i = \beta_1 + \beta_2 \text{ip}_i$.

  A higher $\lambda$ will tend to generate longer strikes. In particular, a unit increase in the explanatory variable will make the average strike $\beta_2$ percent longer. 
\end{frame}

\begin{frame}
  Let's repeat the zoom on one of the elements with the parameterized value of $\lambda_i$

  \begin{displaymath}
  \ln \mathcal{L}_i = \ln \frac{e^{-\lambda_i} \cdot \lambda_i^{y_i}}{{y_i}!}
  =  \ln e^{-\lambda_i} + \ln \lambda_i^{y_i}  - {\ln {y_i}!}  =
\end{displaymath}

\begin{displaymath}
 -\lambda_i + y_i \ln \lambda_i - \ln (y_i) - \ln (y_i - 1) - \ln (y_i - 2) - \cdots - \ln 1
\end{displaymath}

First, rewrite this in $\ln \lambda$ syntax:
\begin{displaymath}
 - e^{\ln \lambda_i} + y_i \ln \lambda_i - \ln (y_i) - \ln (y_i - 1) - \ln (y_i - 2) - \cdots - \ln 1
\end{displaymath}

Now substitute the parameterization, $\ln \lambda_i = \beta_1 + \beta_2 \text{ip}_i$
\begin{displaymath}
 - e^{\beta_1 + \beta_2 \text{ip}_i} + y_i (\beta_1 + \beta_2 \text{ip}_i) - \ln (y_i) - \ln (y_i - 1) - \ln (y_i - 2) - \cdots - \ln 1
\end{displaymath}


\end{frame}


\begin{frame}
  Let's zoom back out and drop the extraneous terms on the right that do not depend on $\beta$ or $\lambda$

  \begin{align}
    \ln \mathcal{L} = \nonumber\\
  - e^{\beta_1 + \beta_2 \text{ip}_1} + y_1 (\beta_1 + \beta_2 \text{ip}_1) \nonumber\\
  - e^{\beta_1 + \beta_2 \text{ip}_2} + y_2 (\beta_1 + \beta_2 \text{ip}_2)\nonumber\\
  - e^{\beta_1 + \beta_2 \text{ip}_2} + y_3 (\beta_1 + \beta_2 \text{ip}_2) \nonumber\\
  \cdots \nonumber\\
  - e^{\beta_1 + \beta_2 \text{ip}_n} + y_n (\beta_1 + \beta_2 \text{ip}_n) \nonumber
  \end{align}


  The maximization is now over two parameters, $\beta_1$ and $\beta_2$,
  instead of over one parameter $\beta$ (or $\lambda$) but as before
  there is some $(\hat{\beta_1}, \hat{\beta_2})$ that makes
  $ \ln \mathcal{L}$ as large as possible. These are the MLE estimates.
  

  
\end{frame}

Here \texttt{beta} is a two-element vector and the expression for $\ln \lambda_i$ includes the value of \texttt{x}$_i$.

\tiny
\begin{verbatim}
poissonll2 <- function(beta) {
    y <- kiefer$strike_duration
    cons <- rep(1, length(kiefer$strike_duration))
    x <- kiefer$ip
    -sum(  -exp(beta[1]*cons + beta[2] * x) + y * (beta[1]*cons + beta[2] * x) )
    }
\end{verbatim}

\end{document}



%%% Local Variables:
%%% mode: latex
%%% TeX-master: t
%%% End:
